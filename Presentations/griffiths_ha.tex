\documentclass{beamer}
\usepackage{amsmath, commath, mathtools}
\usepackage{physics}
\graphicspath{img}

\begin{document}
\title{The Hydrogen Atom}
\author{Daniel Wysocki and Kenny Roffo}
\date{March 26, 2015}

\frame{\titlepage}

\section{The Radial Wave Function}

\frame{
\begin{itemize}
\item Consider a hydrogen atom - a proton, an electron and a radius between them
\item We would like to figure out which energies it is possible for the electron to have by solving for $u(r)$. To do so, we begin with the radial equation:
\begin{displaymath}
-\frac{\hbar^2}{2m}\dod[2]{u}{r} + \left[V + \frac{\hbar^2}{2m}\frac{l(l+1)}{r^2}\right]u=Eu
\end{displaymath}
\item Substituting Coulomb's Law in for $V$ we have
\begin{displaymath}
-\frac{\hbar^2}{2m}\dod[2]{u}{r} + \left[-\frac{e^2}{4\pi r \epsilon_0} + \frac{\hbar^2}{2m}\frac{l(l+1)}{r^2}\right]u=Eu
\end{displaymath}
\end{itemize}
}

\frame{
\begin{itemize}
\item Now we divide both sides by $E$. At this point we will define $\kappa\equiv\frac{\sqrt{-2mE}}{\hbar}$. Note that $E<0$ for bound states, so $\kappa$ is real.
\begin{align*}
u&=-\frac{\hbar^2}{2mE}\dod[2]{u}{r} + \left[-\frac{e^2}{4\pi Er \epsilon_0} + \frac{\hbar^2}{2mE}\frac{l(l+1)}{r^2}\right]u\\
&=\frac{1}{\kappa}\dod[2]{u}{r} + \left[\frac{me^2}{2\pi\epsilon_0 \hbar^2\kappa^2r} - \frac{l(l+1)}{\kappa^2r^2}\right]u\\
&=\dod[2]{u}{\rho} + \left[\frac{\rho_0}{\rho} - \frac{l(l+1)}{\rho^2}\right]u
\end{align*}
\item Here we have defined $\rho\equiv\kappa r$ and $\rho_0\equiv\frac{me^2}{2\pi\epsilon_0\hbar^2\kappa}$
\end{itemize}
}

\frame{
\begin{itemize}
\item Rearranging the equation we have
\begin{displaymath}
\dod[2]{u}{\rho}=\left[1-\frac{\rho_0}{\rho} + \frac{l(l+1)}{\rho^2}\right]u
\end{displaymath}
\item If we consider the equation as $\rho \rightarrow \infty$ (the asymptotic form of the solutions), then we have
\begin{displaymath}
\dod[2]{u}{\rho}=u
\end{displaymath}
\item This ordinary differential equation has general solution
\begin{displaymath}
u(\rho)=Ae^{-\rho}+Be^{\rho}
\end{displaymath}
\item Since $e^\rho \rightarrow \infty$ as $\rho \rightarrow \infty$ we know $B=0$ so when $\rho$ is very large the general solution can be approximated by
\begin{displaymath}
u(\rho)\approx Ae^{-\rho}
\end{displaymath}
\end{itemize}
}

\frame{
\begin{itemize}
\item Now let us consider when $\rho \rightarrow 0$. So long as $l \ne 0$ the centrifugal term will rise far greater than the other terms, so our equation becomes
\begin{displaymath}
\dod[2]{u}{\rho}=\frac{l(l+1)}{\rho^2}u
\end{displaymath}
\item This is also a differential equation, and its general solution is given by
\begin{displaymath}
u(\rho)=C\rho^{l+1}+D\rho^{-l}
\end{displaymath}
\item Snce $\rho^{-l} \rightarrow \infty$ as $\rho \rightarrow 0$ we know $D=0$ so when $\rho$ is very small the general solution can be approximated by
\begin{displaymath}
u(\rho)\approx C\rho^{l+1}
\end{displaymath}
\end{itemize}
}

\frame{
\begin{itemize}
\item We can now synthesize our results in an attempt to get rid of asymptotic behavior
\begin{displaymath}
u(\rho)=v(\rho)\rho^{l+1}e^{-\rho}
\end{displaymath}
\item We are now including a new function $v(\rho)$. By computing $\dod{u}{\rho}$ and $\dod[2]{u}{\rho}$ and plugging into the radial equation we have
\begin{displaymath}
\rho\dod[2]{v}{\rho} + 2(l+1-\rho)\dod{v}{\rho} + [\rho_0-2(l+1)]v = 0
\end{displaymath}
Also, we know from Numerical Analysis techniques that any function can be written as a power series. Thus we know
\begin{displaymath}
v(\rho)=\sum_{j=0}^{\infty}c_j\rho^j
\end{displaymath}
\end{itemize}
}

\frame{
\begin{itemize}
\item Now we must determine the coefficients $(c_0,c_1,c_2,...)$ We see
\begin{align*}
\dod{v}{\rho}&=\sum_{j=0}^{\infty}jc_j\rho^{j-1}=\sum_{j=0}^{\infty}(j+1)c_{j+1}\rho^j\\
\dod[2]{v}{\rho}&=\sum_{j=0}^{\infty}j(j+1)c_{j+1}\rho^{j-1}
\end{align*}
\item Plugging these equations into the radial equation for $v(\rho)$ we have
\begin{multline*}
\sum_{j=0}^{\infty}\left[j(j+1)c_{j+1}\rho^j\right]+2(l+1)\sum_{j=0}^{\infty}\left[(j+1)c_{j+1}\rho^j\right]\\-2\sum_{j=0}^{\infty}\left[jc_j\rho^j\right]+(\rho_0-2(l+1))\sum_{j=0}^{\infty}\left[c_j\rho^j\right]=0
\end{multline*}
\end{itemize}
}

\frame{
\begin{itemize}
\item Dividing through by $p^j$, we have
\begin{displaymath}
j(j+1)c_{j+1}+2(l+1)(j+1)c_{j+1}-2jc_j+(\rho_0-2(l+1))c_j=0
\end{displaymath}
\item And solving for $c_{j+1}$ we have
\begin{displaymath}
c_{j+1}=\frac{2(j+l+1)-\rho_0}{(j+1)(j+2l+2)}c_j
\end{displaymath}
\item Let us consider large values of $j$. Then
\begin{displaymath}
c_{j+1}\approx\frac{2j}{j(j+1)}c_j=\frac{2}{j+1}c_j
\end{displaymath}
\end{itemize}
}

\frame{
\begin{itemize}
\item Now suppose this approximation was exact. Then
\begin{displaymath}
c_j=\frac{2^j}{j!}c_0
\end{displaymath}
\item and then we would have
\begin{displaymath}
v(\rho)=c_0\sum_{j=0}^{\infty}\frac{2^j}{j!}p^j=c_0e^{2\rho}
\end{displaymath}
\item But this would mean
\begin{displaymath}
u(\rho)=c_0\rho^{l+1}e^\rho
\end{displaymath}
\item so $u(\rho)$ displays asymptotic behavior, which is what we were trying to get rid of.
\end{itemize}
}

\frame{
\begin{itemize}
\item It now seems that there is only one way to deal with this issue - The series must be finite.
\item There must exist a maximum $j$ such that
\begin{displaymath}
c_{j_{max}+1}=0
\end{displaymath}
\item This implies 
\begin{displaymath}
2(j_{max}+l+1)=\rho_0
\end{displaymath}
\item We now define the \textbf{principal quantum number}, $n$, to be
\begin{displaymath}
n\equiv j_{max}+l+1
\end{displaymath}
\end{itemize}
}

\frame{
\begin{itemize}
\item Note that
\begin{displaymath}
\rho_0=2n
\end{displaymath}
\item Recall that energy depends on $\rho_0$
\begin{displaymath}
E=-\frac{\hbar^2\kappa^2}{2m}=-\frac{me^4}{8\pi^2\epsilon_0^2\hbar^2\rho_0^2}
\end{displaymath}
\item Therefore we have at last determined the allowed energies to be
\begin{displaymath}
E_n=-\left[\frac{m}{2\hbar^2}\left(\frac{e^2}{4\pi\epsilon_0}\right)^2\right]\frac{1}{n^2}=\frac{E_1}{n^2} \text{\hspace{0.2 in} where $n\in\{1,2,3,...\}$}
\end{displaymath}
\end{itemize}
}

\frame{
\begin{itemize}
\item Recall
\begin{displaymath}
\rho_0\equiv\frac{me^2}{2\pi\epsilon_0\hbar^2\kappa} \text{\hspace{0.5 in}} \rho_0=2n
\end{displaymath}
\item Combining these we have
\begin{displaymath}
\kappa=\left(\frac{me^2}{4\pi\epsilon_0\hbar^2}\right)\frac{1}{n}=\frac{1}{an}
\end{displaymath}
\item where $a$ is the Bohr Radius
\begin{displaymath}
a=\frac{4\pi\epsilon_0\hbar^2}{me^2}\approx 5.29 \times 10^{-11} \text{m}
\end{displaymath}
\item Recalling the definition of $\rho$ we also now see
\begin{displaymath}
\rho=\frac{r}{an}
\end{displaymath}
\end{itemize}
}

\frame{
\begin{itemize}
\item So far we have three quantum numbers, $n$, $l$, and $m$. The spatial wave functions for hydrogen are labeled
\begin{displaymath}
\psi_{nlm}(r,\theta,\phi)=R_{nl}(r)Y_l^m(\theta,\phi)
\end{displaymath}
\item Recall from \textbf{section 4.1}
\begin{displaymath}
R_{nl}(r)=\frac{1}{r}\rho^{l+1}e^{-\rho}v(\rho)
\end{displaymath}
\item The ground state is the case where $n=1$. Using our best approximations for the physical constants we have
\begin{displaymath}
E_1=-\left[\frac{m}{2\hbar}\left(\frac{e^2}{4\pi\epsilon_0}\right)^2\right]=-13.6 \text{eV}
\end{displaymath}
\item As expected, the binding energy, or the energy necessary to ionize a a Hydrogen atom in the ground state, is 13.6 eV
\end{itemize}
}

\frame{
\begin{itemize}
\item Consider $\psi_{100}(r,\theta,\phi)=R_{10}(r)Y_0^0(\theta,\phi)$.
\begin{displaymath}
R_{10}(r)=\frac{c_0}{a}e^{-r/a}
\end{displaymath}
\item Normalizing we have
\begin{displaymath}
\int_{0}^{\infty}|R_{10}|^2r^2\dif{r}=\frac{|c_0|^2}{a^2}\int_{0}^{\infty}e^{-2r/a}r^2\dif{r}=|c_0|^2\frac{a}{4}=1
\end{displaymath}
\item Thus $c_0=\frac{2}{\sqrt{a}}$. Also, $Y_0^0=\frac{1}{\sqrt{4\pi}}$, so the ground state of hydrogen is
\begin{displaymath}
\psi_{100}(r,\theta,\phi)=\frac{1}{\sqrt{\pi a^3}}e^{-r/a}
\end{displaymath}
\end{itemize}
}

%Left off Half-way down page 153
%\frame{
%\begin{itemize}
%\end{itemize}
%}

\frame{Thank you!}
\end{document}
