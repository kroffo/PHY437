\documentclass[t]{beamer}
\usetheme{Warsaw}
\usepackage{amsmath,amsthm,amssymb,hyperref,physics}
\usepackage{,mathtools,graphicx,commath}
\definecolor{beamer@blendedgreen}{rgb}{0.0, 0.4, 0.0}
\definecolor{beamer@blendedgold}{rgb}{1.0, 0.8, 0.0}

\setbeamercolor*{palette primary}{use=structure,fg=black,bg=beamer@blendedgold!90}
\setbeamercolor*{palette quaternary}{fg=beamer@blendedgold,bg=beamer@blendedgreen!90!beamer@blendedgold}

\title{Two-Particle Systems}
\author{Dylan Mcintyre and Kenny Roffo}
\date{April 23, 2015}
\institute{SUNY Oswego}

\begin{document}

\frame{
\titlepage
}

\frame{
\frametitle{Changing from 1 to 2 Particles}
\begin{itemize}
\item For a single particle, the wave function is
$$\Psi(\vb{r},t)$$
\pause
\item Adding a second particle to the system, we write
$$\Psi(\vb{r}_1,\vb{r}_2,t)$$
\pause
\item To use the Schr\"{o}dinger equation we must use know the Hamiltonian:
\begin{displaymath}
H = -\frac{\hbar^2}{2m_1}\grad_1^2 - \frac{\hbar^2}{2m_2}\grad_2^2
     + V(\vb{r}_1,\vb{r}_2,t)
\end{displaymath}
\end{itemize}
}

\frame{
\frametitle{Changing from 1 to 2 Particles}
\begin{itemize}
\item The probability of finding each particle in a given volume $\dif{^3\vb{r}}_i$ is 
$$\int|\Psi(\vb{r}_1,\vb{r}_2,t)|^2\dif{^3\vb{r}_1}\dif{^3\vb{r}_2}=1$$
\pause
\item For time independent potentials, we have the general solution:
$$\Psi(\vb{r}_1,\vb{r}_2,t)=\psi(\vb{r}_1,\vb{r}_2)e^{-iEt/\hbar}$$
\pause
\item Here, $\psi$ satisfies the time-independent Schr\"{o}dinger wave equation:
\begin{displaymath}
-\frac{\hbar^2}{2m_1}\grad_1^2\psi - \frac{\hbar^2}{2m_2}\grad_2^2\psi + V\psi = E\psi
\end{displaymath}
\end{itemize}
}

\frame{
\frametitle{The Difference Between 2 Particles}
\begin{itemize}
\item Consider two objects. One in the state $\psi_a$, the other in the state $\psi_b$
\pause
\item Classically, we can tell these objects from one another
\pause
\item In Quantum Mechanics, this is not the case
\pause
\item Particles are inherently identical; there is no way to tell the difference between two electrons
\pause
\item With particles, we can only know that one of the two is in state $\psi_a$, and the other is in state $\psi_b$, but we cannot tell them apart
\pause
\item We represent this by a wave function which doesn't assume either particle is in a particular state:
\begin{displaymath}
\psi\pm(\vb{r}_1,\vb{r}_2)=A\left[\psi_a(\vb{r}_1)\psi_b(\vb{r}_2)\pm\psi_b(\vb{r}_1)\psi_a(\vb{r}_2)\right]
\end{displaymath}
\end{itemize}
}

\frame{
\frametitle{Bosons and Fermions}
\begin{displaymath}
\psi\pm(\vb{r}_1,\vb{r}_2)=A\left[\psi_a(\vb{r}_1)\psi_b(\vb{r}_2)\pm\psi_b(\vb{r}_1)\psi_a(\vb{r}_2)\right]
\end{displaymath}
\begin{itemize}
\item Choosing $+$ or $-$ we have two types of particles
\pause
\item \textbf{Bosons} are those particles for which we use $+$
\item \textbf{Fermions} are those particles for which we use $-$
\pause
\item It works out that Bosons are particles with integer spin, and Fermions are particles with half-integer spin.
\pause
\item As a consequence, we see we cannot have two Fermions cannot occupy the same state else $\psi_a=\psi_b$ and
\begin{displaymath}
\psi_-(\vb{r}_1,\vb{r}_2)=A\left[\psi_a(\vb{r}_1)\psi_b(\vb{r}_2)-\psi_a(\vb{r}_1)\psi_b(\vb{r}_2)\right] = 0
\end{displaymath}
\item This result is the famous \emph{Pauli Exclusion Principle}
\end{itemize}
}

\frame{
\frametitle{DYLAN'S WONDERFUL STUFF}
\begin{itemize}
\item Dylan
\item Make
\item Your
\item Slides
\end{itemize}
}

\frame{
Thank you!
}
\end{document}
