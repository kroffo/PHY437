\documentclass{scrartcl}
\usepackage{amsmath,amssymb}
\usepackage{commath,physics}
\setkomafont{disposition}{\normalfont\bfseries}

\title{Advanced Quantum Mechanics}
\subtitle{Homework 5: 4.1, 4.2, 4.16, 4.17}
\author{Kenny Roffo}
\date{Due March 26}

\begin{document}

\maketitle

\textbf{4.1 a)} Work out all of the canonical commutation relations for components of the operators $\vb{r}$ and $\vb{p}$:
\begin{align*}
[x,y]=[y,x]&=xy-yx=xy-xy=0\\
[x,z]=[z,x]&=xz-zx=xz-xz=0\\
[y,z]=[z,y]&=yz-zy=yz-yz=0
\end{align*}\\
Since $r_x=x$, $r_y=y$ and $r_z=z$ this implies $[r_i,r_j]=0$ where $i$ and $j$ are $x$, $y$ or $z$. ($[r_i,r_i]=0$ is trivial since the commutator of anything with itself is always 0).\\

Recall that $p_x=\frac{\hbar}{i} \pd{}{x}$. Letting $f$ be a test function we have:
\begin{align*}
[p_x,p_y]&=\left[\left(\frac{\hbar}{i} \dpd{}{x}\right)\left(\frac{\hbar}{i} \dpd{}{y}\right)-\left(\frac{\hbar}{i} \dpd{}{y}\right)\left(\frac{\hbar}{i} \dpd{}{x}\right)\right]f\\
&=-\hbar^2\left(\dpd{}{x}\dpd{}{y}f-\dpd{}{y}\dpd{}{x}f\right)
\end{align*}
But we know from calculus that for any function $g$, $\pd{}{x}\pd{}{y}g = \pd{}{y}\pd{}{x}g$, so this implies $[p_x,p_y]=0$. Generalizing this tells us that $[p_i,p_j]=0$ when $i$ and $j$ are $x$, $y$ or $z$.\\

Now consider $[x,p_y]$. Applying a test function $f$ we have:
\begin{align*}
[x,p_y]=-[p_y,x]=[x,\frac{\hbar}{i} \dpd{}{y}]&=\left[x\frac{\hbar}{i} \dpd{}{y} - \frac{\hbar}{i} \dpd{}{y}x\right]f\\
&=x\frac{\hbar}{i} \dpd{f}{y}-\frac{\hbar}{i} \dpd{x\cdot f}{y}\\
&=x\frac{\hbar}{i} \dpd{f}{y}-\left(x\frac{\hbar}{i} \dpd{f}{y} + f\frac{\hbar}{i} \dpd{x}{y}\right)\\
&=-f\frac{\hbar}{i}\dpd{x}{y} \text{\hspace{1.5 in} but } \dpd{x}{y}=0 \text{ so... }\\
&=0
\end{align*}

So $[r_i,p_j]=0$ when $i$ and $j$ are distinct and $x$, $y$ or $z$. But what about when $i=j$?
\begin{align*}
[x,p_x]=-[p_x,x]&=[x,\frac{\hbar}{i} \dpd{}{x}]\\
&=\left[x\frac{\hbar}{i} \dpd{}{x} - \frac{\hbar}{i} \dpd{}{x}x\right]f\\
&=x\frac{\hbar}{i} \dpd{f}{x}-\frac{\hbar}{i} \dpd{x\cdot f}{x}\\
&=x\frac{\hbar}{i} \dpd{f}{x}-\left(x\frac{\hbar}{i} \dpd{f}{x} + f\frac{\hbar}{i} \dpd{x}{x}\right)\\
&=-f\frac{\hbar}{i}
\end{align*} 
Now removing the test function, we see $[r_i,p_i]=-\frac{\hbar}{i}$.\\
Synthesizing these results we see that the function is 0 except when $i$ and $j$ are 0. We can thus represent this by one function using the \emph{kronecker delta function}:
\begin{displaymath}
[r_i,p_j]=-[p_i,r_j]=-\frac{\hbar}{i}\delta_{ij}=i\hbar\delta_{ij}
\end{displaymath}\\

\textbf{4.1 b)} We must show that $\od{}{t}\expval{r}=\frac{1}{m}\expval{p}$. We must first check for the x-component. From Equation 3.71 we know:
\begin{displaymath}
\dpd{\expval{Q}}{t}=\frac{i}{\hbar}\expval{[\hat{H},\hat{Q}]}+\expval{\dpd{\hat{Q}}{t}}
\end{displaymath}
Now
\begin{align*}
[\hat{H},\hat{x}]&=[\frac{p^2}{2m}+V,x]\\
&=\frac{1}{2m}[p_x^2+p_y^2+p_z^2,x] + [V,x]\\
&=\frac{1}{2m}\left([p_x^2,x]+[p_y^2,x]+[p_z^2,x]\right) + 0\\
&=\frac{1}{2m}[p_x^2,x] \text{\hfill by \textbf{4.1 a}}\\
&=\frac{1}{2m}\left(p_x[p_x,x]+[p_x,x]p_x\right)\\
&=\frac{1}{2m}\left(p_x(-i\hbar\delta_{xx})+p_x(-i\hbar\delta_{xx})\right) \text{\hfill by \textbf{4.1 a}}\\
&=\frac{1}{2m}\left(-2p_xi\hbar\right)\\
&=\frac{-p_xi\hbar}{m}\\
&=\frac{p_x\hbar}{im}
\end{align*}
By Equation 3.71 we have 
\begin{displaymath}
\dod{\expval{x}}{t}=\frac{i}{\hbar}\expval{[\hat{H},\hat{x}]}+\expval{\dpd{\hat{x}}{t}}\\
\end{displaymath}
And now that we have $[\hat{H},\hat{x}]$ we see:
\begin{align*}
\dod{\expval{x}}{t}&=\frac{i}{\hbar}\expval{\frac{p_x\hbar}{im}}+0 \text{\hspace{1 in} since $\pd{x}{t}=0$}\\
&=\frac{i}{\hbar}\frac{\hbar}{im}\expval{p_x}\\
&=\frac{1}{m}\expval{p_x}
\end{align*}
Thus $\od{}{t}\expval{x}=\frac{1}{m}\expval{p_x}$. Applying the same steps to $y$ and $x$ we see:
\begin{displaymath}
\dod{}{t}\expval{y}=\frac{1}{m}\expval{p_y} \text{\hspace{1 in}} \dod{}{t}\expval{z}=\frac{1}{m}\expval{p_z}
\end{displaymath}
Therefore, synthesizing these results we have:
\begin{displaymath}
\dod{}{t}\expval{\vb{r}}=\frac{1}{m}\expval{\vb{p}}
\end{displaymath}

Now we must show that $\od{}{t}\expval{\vb{p}}=\expval{\grad V}$. Again we will start with the x-component:
\begin{align*}
[\hat{H},\hat{p}]&=[\frac{p^2}{2m}+V,p_x]\\
&=\frac{1}{2m}[p_x^2+p_y^2+p_z^2,p_x]+[V,p_x]\\
&=\frac{1}{2m}\left([p_x^2,p_x]+[p_y^2,p_x]+[p_z^2,p_x]\right)+[V,p_x]\\
&=[V,p_x] \text{\hspace{1.5 in} by \textbf{4.1 a}}\\
&=[V,-i\hbar\dpd{}{x}]\\
\end{align*}
We will use a test function $f$ to calculate this commutator:
\begin{align*}
[V,p_x]=[V,-i\hbar\dpd{}{x}]f&=-Vi\hbar\dpd{f}{x}+i\hbar\dpd{V\cdot f}{x}\\
&=-Vi\hbar\dpd{f}{x}+Vi\hbar\dpd{f}{x}+fi\hbar\dpd{V}{x}\\
&=fi\hbar\dpd{V}{x}
\end{align*}
Now removing the test function we see $[\hat{H},\hat{p_x}]=i\hbar\dpd{V}{x}$. Now we can calculate $\od{}{t}\expval{p_x}$:
\begin{align*}
\dod{}{t}\expval{p_x}&=\frac{i}{\hbar}\expval{[\hat{H},\hat{p_x}]}+\expval{\dpd{\hat{p_x}}{t}}\\
&=\frac{i}{\hbar}\expval{i\hbar\dpd{V}{x}}+0 \text{\hspace{1.5 in} since $\dpd{\hat{p_x}}{t}=0$}\\
&=i^2\expval{\dpd{V}{x}}\\
&=\expval{-\dpd{V}{x}}
\end{align*}
Applying the same steps to $y$ and $z$ reveal that:
\begin{displaymath}
\dod{}{t}\expval{p_y}=\expval{-\dpd{V}{y}} \text{\hspace{1.5 in}} \dod{}{t}\expval{p_z}=\expval{-\dpd{V}{z}}
\end{displaymath}
Synthesizing these results we see:
\begin{displaymath}
\dod{}{t}\expval{\vb{p}}=\expval{-\left(\dpd{}{x}+\dpd{}{y}+\dpd{}{z}\right)V}
\end{displaymath}
Thus we have
\begin{displaymath}
\dod{}{t}\expval{\vb{p}}=\expval{-\grad V}
\end{displaymath}

\textbf{4.1 c)} Formulate Heisenberg's uncertainty principle in three dimensions.
From \textbf{section 3.5} we know the uncertainty principle for two operators $\hat{A}$ and $\hat{B}$ is:
\begin{displaymath}
\sigma_A\sigma_B \ge \frac{1}{2i}\expval{[\hat{A},\hat{B}]}
\end{displaymath}
by \textbf{4.1 a} we know the commutators of the different combinations among $x$, $y$, $z$, $p_x$, $p_y$ and $p_z$. Thus to find the uncertainty principle for these combinations we must simply plug those commutators into this equation and calculate.
\begin{align*}
\sigma_x\sigma_{p_x} &\ge \frac{1}{2i}i\hbar\delta_{xx}\\
&=\frac{i\hbar}{2i}\\
&=\frac{\hbar}{2}
\end{align*}
And generalizing this implies that $\sigma_i,\sigma_{p_i}\ge\frac{\hbar}{2}$ when $i$ is one of $x$, $y$ or $z$.\\
Also, for all other combinations the commutator is 0, so the uncertainty principle for all other combinations is
\begin{displaymath}
\sigma_A\sigma_B \ge 0
\end{displaymath}
when $A$ and $B$ are distinct and each one of $x$, $y$, $z$, $p_x$, $p_y$ and $p_z$. This means there is no restriction on the uncertainty of these combinations (thought they cannot be negative).\pagebreak

\textbf{4.2} Use seperation of variables in \emph{cartesian} coordinates to solve the infinite \emph{cubical} well:
\begin{displaymath}
V(x,y,z)=
\begin{cases}
0 & \text{if $x$, $y$, $z$ are all between 0 and $a$}\\
\infty & \text{otherwise}
\end{cases}
\end{displaymath}
\textbf{a)} Find the stationary states, and the corresponding energies.
The time-independent schrodinger equation for three dimensions is
\begin{displaymath}
-\frac{\hbar^2}{2m}\grad^2\psi+V\psi=E\psi
\end{displaymath}
Plugging in $V$, when the particle is in the box, we have
\begin{displaymath}
-\frac{\hbar^2}{2m}\grad^2\psi=-\frac{\hbar^2}{2m}\left(\dpd[2]{\psi}{x}+\dpd[2]{\psi}{y}+\dpd[2]{\psi}{z}\right)=E\psi
\end{displaymath}
Assuming the wave function is seperable, it can be written as
\begin{displaymath}
\psi(x,y,z)=X(x)Y(y)Z(z)
\end{displaymath}
Pluggin in we have
\begin{displaymath}
-\frac{\hbar^2}{2m}\left(YZ\dod[2]{X}{x}+XZ\dod[2]{Y}{y}+XY\dod[2]{Z}{z}\right)=E\psi
\end{displaymath}
Dividing by $\psi$ we have
\begin{displaymath}
\frac{1}{X}\dod[2]{X}{x}+\frac{1}{Y}\dod[2]{Y}{y}+\frac{1}{Z}\dod[2]{Z}{z}=-\frac{2m}{\hbar^2}E
\end{displaymath}
This implies that each of the terms on the left side must be constant. Letting
\begin{displaymath}
C_x + C_y + C_z =-\frac{2m}{\hbar^2}E
\end{displaymath}
we have that
\begin{displaymath}
\begin{split}
\frac{1}{X}\dod[2]{X}{x} = -C_x \hspace{0.75 in}
\end{split}
\begin{split}
\frac{1}{Y}\dod[2]{Y}{y} = -C_y \hspace{0.75 in}
\end{split}
\begin{split}
\frac{1}{Z}\dod[2]{Z}{z} = -C_z
\end{split}
\end{displaymath}
But this implies
\begin{displaymath}
\begin{split}
\frac{1}{X}\dod[2]{X}{x} + C_x = 0 \hspace{0.75 in}
\end{split}
\begin{split}
\frac{1}{Y}\dod[2]{Y}{y} + C_y = 0 \hspace{0.75 in}
\end{split}
\begin{split}
\frac{1}{Z}\dod[2]{Z}{z} + C_z = 0
\end{split}
\end{displaymath}
These equations have general solutions:
\begin{align*}
X(x)&=A_x\sin(C_xx)+B_x\cos(C_xx)\\
Y(y)&=A_y\sin(C_yy)+B_y\cos(C_yy)\\
Z(z)&=A_z\sin(C_zz)+B_z\cos(C_zz)\\
\end{align*}
The boundary conditions state these functions must be 0 at $x,y,z=0$. Thus the $B$ coefficients must all be equal to 0, and so
\begin{align*}
X(x)&=A_x\sin(C_xx)\\
Y(y)&=A_y\sin(C_yy)\\
Z(z)&=A_z\sin(C_zz)\\
\end{align*}
Also, at $x,y,z=a$ these functions must be 0, so $\sin(C_xa)=0$. $\sin(\theta)=0$ when $\theta=n\pi$ so we have
\begin{align*}
C_x=\frac{n_x\pi}{a}\\
C_y=\frac{n_y\pi}{a}\\
C_z=\frac{n_z\pi}{a}\\
\end{align*}
for integers $n_x, n_y, n_z$.

We now have
\begin{displaymath}
\psi(x,y,z)=X(x)Y(y)Z(z)=A_x\sin(C_xx)A_y\sin(C_yy)A_z\sin(C_zz)
\end{displaymath}
Normalization of the function is shown for $A_x$:
\begin{align*}
1&=\int_{0}^{a}\left[A_x\sin(C_xx)\right]^*\left[A_x\sin(C_xx)\right]\dif{x}\\
&=\int_{0}^{a}A_x^2\sin^2(C_xx)\dif{x}\\
&=\frac{a}{2}A_x^2\\
\implies
 A_x&=\sqrt{\frac{2}{a}}
\end{align*}
By the same process we see $A_y=A_z=\sqrt{\frac{2}{a}}$
Therefore we now have the normalized equation for the energies of stationary states of a particle in a three-dimensional box.
\begin{displaymath}
-\frac{2m}{\hbar^2}E=\sqrt{\frac{8}{a^3}}\sin(\frac{n_x\pi}{a}x)\sin(\frac{n_y\pi}{a}y)\sin(\frac{n_z\pi}{a}z)
\end{displaymath}
\begin{displaymath}
  \implies E= -\sqrt{\frac{2\hbar^4}{a^3m^2}}\sin(\frac{n_x\pi}{a}x)\sin(\frac{n_y\pi}{a}y)\sin(\frac{n_z\pi}{a}z)
\end{displaymath}
The Energy eigenvalues are now given by
\begin{displaymath}
E = \frac{\pi^2\hbar^2}{2ma^2}(n_x^2+n_y^2+n_z^2)
\end{displaymath}\\\\

\textbf{b)} Call the distinct energies $E_1, E_2, E_3,...$ in order of increasing energy. Find $E_1,E_2,...,E_6$ and determine their degeneracies:

For $E_n$ the sum of $n_x+n_y+n_z-2$ must equal $n$. Thus for $E_1$ there is only one such state, the ground state. However, for $E_2$ there are three possible ways to choose $\{n_x,n_y,n_z\}$ : $\{2,1,1\}, \{1,2,1\}, \{1,1,2\}$ so $E_2$ has a degeneracy of 3. Via the same process we see the degeneracies are:
\begin{align*}
E_1 = \frac{3\pi^2\hbar^2}{2ma^2} &: d=1\ \{1,1,1\}\\
E_2 = \frac{6\pi^2\hbar^2}{2ma^2} &: d=3\ \{2,1,1\},\{1,2,1\},\{1,1,2\}\\
E_3 = \frac{9\pi^2\hbar^2}{2ma^2} &: d=9\ \{1,2,2\},\{2,1,2\},\{2,2,1\}\\
E_4 = \frac{11\pi^2\hbar^2}{2ma^2} &: d=3\ \{3,1,1\},\{1,3,1\},\{1,1,3\}\\
E_5 = \frac{12\pi^2\hbar^2}{2ma^2} &: d=1\ \{2,2,2\}\\
E_6 = \frac{14\pi^2\hbar^2}{2ma^2} &: d=6\ \{3,2,1\},\{3,1,2\},\{2,3,1\},\{1,3,2\},\{1,2,3\},\{2,1,3\}
\end{align*}\\\\

\textbf{c)} What is the degeneracy of $E_{14}$, and why is this case interesting?
\begin{displaymath}
E_{14}=\frac{27\pi^2\hbar^2}{2ma^2} : d=4\ \{3,3,3\},\{5,1,1\},\{1,5,1\},\{1,1,5\}
\end{displaymath}
This case is very interesting as in all states before hand there was really only one way to choose $n_x,n_y,n_z$ (then they could be shuffled around). However, at $E_{14}$ there are \emph{two} different ways to choose this. Either, all of them are 3, or one is 5 and the others are 1.\pagebreak

\textbf{4.16)} A hydrogenic atom consists of a single electron orbiting a nucleus with $Z$ protons. Determine the Bohr energies, $E_n(Z)$, the binding energy $E_1(Z)$, and the Bohr radius $a(Z)$, and the Rydberg constant $R(Z)$ for a hydrogenic atom. \\

The allowed energies for an electron in a Hydrogen atom are
\begin{displaymath}
E_n=-\left[\frac{m}{2\hbar^2}\left(\frac{e^2}{4\pi\epsilon_0}\right)^2\right]\frac{1}{n^2}
\end{displaymath}
For a Hydrogenic atom, we know $e^2\rightarrow Ze^2$, so the allowed energies for an electron in a Hydrogenic atom are
\begin{displaymath}
E_n(Z)=-\left[\frac{m}{2\hbar^2}\left(\frac{Ze^2}{4\pi\epsilon_0}\right)^2\right]\frac{1}{n^2}=Z^2\frac{E_1}{n^2}
\end{displaymath}
So then the binding energy of a hydrogenic atom is
\begin{displaymath}
E_1(Z)=Z^2E_1=-13.6Z^2
\end{displaymath}
The Bohr Radius of Hydrogen is
\begin{displaymath}
a_0\equiv\frac{4\pi\epsilon_0\hbar^2}{me^2}
\end{displaymath}
Applying the conversion for a Hydrogenic atom we have
\begin{displaymath}
a(Z)=\frac{4\pi\epsilon_0\hbar^2}{Zme^2}=\frac{a_0}{Z}
\end{displaymath}
The Rydberg constant for Hydrogen is
\begin{displaymath}
R_H\equiv\frac{m}{4\pi c\hbar^3}\left(\frac{e^2}{4\pi\epsilon_0}\right)^2
\end{displaymath}
Once again we convert this to find the Rydberg Constant for any Hydrogenic atom:
\begin{displaymath}
R(Z)=\frac{m}{4\pi c\hbar^3}\left(\frac{Ze^2}{4\pi\epsilon_0}\right)^2=Z^2R_H
\end{displaymath}
The wavelengths for the Lyman series of Hydrogen are in the Ultraviolet range. $R_{He^+}=4.388\times10^{-7}$ and $R_{Li^{++}}=9.873\times10^{-7}$This is not that significant of a change from $R_H$, and it comes as no surprise that the Lyman series for these two ions still lies in the UV range.\pagebreak

\textbf{4.17)} Consider the earth-sun system as a gravitational analog to the hydrogen atom.
\textbf{a)} What is the potential energy function?
By comparison the potential energy function would be
\begin{displaymath}
V(r)=-\frac{GmM}{r}
\end{displaymath}
where $G$ is the Universal Gravitational Constant.
\\
\textbf{b)} What is the ``Bohr Radius'', $a_g$, for this system?
\begin{displaymath}
a_g=\frac{\hbar^2}{Gm^2M}=\frac{(1.05457\times10^{-34})^2}{(6.67\times10^{-11})(6\times10^{24})^2(2\times10^{30})}\approx 2.316\times10^{-138}
\end{displaymath}
\\
\textbf{c)}Write down the gravitational ``Bohr Formula'', and, by equating $E_n$ to the classical energy of a planet in a circular orbit of radius $r_0$, show that $n=\sqrt{r_0/a_g}$. From this, estimate the quantum number $n$ of the Earth.
\begin{displaymath}
E_n=-\left[\frac{m}{2\hbar^2}\left(GmM\right)^2\right]\frac{1}{n^2}=-2\frac{GmM}{r_0}
\end{displaymath}
Solving this equation for $n$ we have:
\begin{align*}
n^2&=\left[\frac{m}{2\hbar^2}\left(GmM\right)\right]2r_0\\
n^2&=\frac{Gm^2M}{\hbar^2}r_0\\
n^2&=\frac{r_0}{a_g}\\
n&=\sqrt{\frac{r_0}{a_g}}
\end{align*}
\\
The principle quantum number of the Earth is then
\begin{displaymath}
n=\sqrt{\frac{1.5\times10^{11}}{2.31575\times10^{-138}}}=2.54\times10^{74}
\end{displaymath}
\\
\textbf{d)} Suppose the earth made a transition to the lower level ($n-1$). How much energy would be released? What would the wavelength of the emitted photon be?\\

%The Rydberg Constant for Earth would be
%\begin{displaymath}
%R_{Earth}=\frac{m}{4\pi c\hbar^3}G^2m^2M^2\approx 9.67372\times10^{138}
%\end{displaymath}
%Now we have
%\begin{align*}
%\lambda=\frac{1}{R\left(\frac{1}{n_f^2}-\frac{1}{n_i^2}\right)}
%&\approx\frac{1}{(9.67372\times10^{138})\left(\frac{1}{(2.54\times10^{74}-1)^2}-\frac{1}{(2.54\times10^{74})^2}\right)}
%&=
%\end{align*}
\begin{align*}
E_\gamma&=\left[\frac{m}{2\hbar^2}\left(GmM\right)^2\right]\left(\frac{1}{n_f^2}-\frac{1}{n_i^2}\right)\\
&=2.1\times10^{-41} \text{ joules}
\end{align*}
Now we calculate the wavelength using $\lambda=\frac{hc}{E_\gamma}$
\begin{align*}
\lambda&\approx 9.496\times10^{15} \text{ m}\\
&\approx 1 \text{lightyear}
\end{align*}
This is absolutely astonishing!
\end{document}
